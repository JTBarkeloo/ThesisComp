%-------------------------------------------------------------------------------
% This file provides a skeleton ATLAS note.
% \pdfinclusioncopyfonts=1
% This command may be needed in order to get \ell in PDF plots to appear. Found in
% https://tex.stackexchange.com/questions/322010/pdflatex-glyph-undefined-symbols-disappear-from-included-pdf
%-------------------------------------------------------------------------------
% Specify where ATLAS LaTeX style files can be found.
\newcommand*{\ATLASLATEXPATH}{latex/}
% Use this variant if the files are in a central location, e.g. $HOME/texmf.
% \newcommand*{\ATLASLATEXPATH}{}
%-------------------------------------------------------------------------------
\documentclass[NOTE, atlasdraft=true, texlive=2016, UKenglish]{\ATLASLATEXPATH atlasdoc}
% The language of the document must be set: usually UKenglish or USenglish.
% british and american also work!
% Commonly used options:
%  atlasdraft=true|false This document is an ATLAS draft.
%  texlive=YYYY          Specify TeX Live version (2016 is default).
%  coverpage             Create ATLAS draft cover page for collaboration circulation.
%                        See atlas-draft-cover.tex for a list of variables that should be defined.
%  cernpreprint          Create front page for a CERN preprint.
%                        See atlas-preprint-cover.tex for a list of variables that should be defined.
%  NOTE                  The document is an ATLAS note (draft).
%  PAPER                 The document is an ATLAS paper (draft).
%  CONF                  The document is a CONF note (draft).
%  PUB                   The document is a PUB note (draft).
%  BOOK                  The document is of book form, like an LOI or TDR (draft)
%  txfonts=true|false    Use txfonts rather than the default newtx
%  paper=a4|letter       Set paper size to A4 (default) or letter.

%-------------------------------------------------------------------------------
% Extra packages:
\usepackage{\ATLASLATEXPATH atlaspackage}
% Commonly used options:
%  biblatex=true|false   Use biblatex (default) or bibtex for the bibliography.
%  backend=bibtex        Use the bibtex backend rather than biber.
%  subfigure|subfig|subcaption  to use one of these packages for figures in figures.
%  minimal               Minimal set of packages.
%  default               Standard set of packages.
%  full                  Full set of packages.
%-------------------------------------------------------------------------------
% Style file with biblatex options for ATLAS documents.
\usepackage{\ATLASLATEXPATH atlasbiblatex}
% Commonly used options:
%  backref=true|false    Turn on or off back references in the bibliography.

% Package for creating list of authors and contributors to the analysis.
\usepackage{\ATLASLATEXPATH atlascontribute}

% Useful macros
\usepackage{\ATLASLATEXPATH atlasphysics}
% See doc/atlas_physics.pdf for a list of the defined symbols.
% Default options are:
%   true:  journal, misc, particle, unit, xref
%   false: BSM, heppparticle, hepprocess, hion, jetetmiss, math, process,
%          other, snippets, texmf
% See the package for details on the options.

% Files with references for use with biblatex.
% Note that biber gives an error if it finds empty bib files.
%\addbibresource{FCNCtqaINT.bib}
%\addbibresource{bib/ATLAS.bib}
%\addbibresource{bib/CMS.bib}
%\addbibresource{bib/ConfNotes.bib}
%\addbibresource{bib/PubNotes.bib}

% Paths for figures - do not forget the / at the end of the directory name.
\graphicspath{{logos/}{figures/}}

% Add you own definitions here (file FCNCtqaINT-defs.sty).
\usepackage{FCNCtqaINT-defs}

%-------------------------------------------------------------------------------
% Generic document information
%-------------------------------------------------------------------------------

% Title, abstract and document 
%-------------------------------------------------------------------------------
% This file contains the title, author and abstract.
% It also contains all relevant document numbers used for an ATLAS note.
%-------------------------------------------------------------------------------

% Title
\AtlasTitle{AtlasTitle: Bare bones ATLAS document}

% Draft version:
% Should be 1.0 for the first circulation, and 2.0 for the second circulation.
% If given, adds draft version on front page, a 'DRAFT' box on top of each other page, 
% and line numbers.
% Comment or remove in final version.
\AtlasVersion{0.1}

% Abstract - % directly after { is important for correct indentation
\AtlasAbstract{%
  This is a bare bones ATLAS document. Put the abstract for the document here.
}

% Author - this does not work with revtex (add it after \begin{document})
%\author{Jason Barkeloo}

% Authors and list of contributors to the analysis
% \AtlasAuthorContributor also adds the name to the author list
% Include package latex/atlascontribute to use this
% Use authblk package if there are multiple authors, which is included by latex/atlascontribute
 \usepackage{authblk}
% Use the following 3 lines to have all institutes on one line
 \makeatletter
 \renewcommand\AB@affilsepx{, \protect\Affilfont}
 \makeatother
 \renewcommand\Authands{, } % avoid ``. and'' for last author
 \renewcommand\Affilfont{\itshape\small} % affiliation formatting
 \AtlasAuthorContributor{Jason Barkeloo}{a}{All Aspects}
% \AtlasAuthorContributor{Second AtlasAuthorContributor}{b}{Author's contribution.}
% \AtlasAuthorContributor{Third AtlasAuthorContributor}{a}{Author's contribution.}
% \AtlasContributor{Fourth AtlasContributor}{Contribution to the analysis.}
% \author[a]{Jason Barkeloo}
%% \author[a]{Second Author}
% \author[b]{Third Author}
 \affil[a]{University of Oregon}
%% \affil[b]{Another Institution}

% If a special author list should be indicated via a link use the following code:
% Include the two lines below if you do not use atlasstyle:
% \usepackage[marginal,hang]{footmisc}
% \setlength{\footnotemargin}{0.5em}
% Use the following lines in all cases:
% \usepackage{authblk}
% \author{The ATLAS Collaboration%
% \thanks{The full author list can be found at:\newline
%   \url{https://atlas.web.cern.ch/Atlas/PUBNOTES/ATL-PHYS-PUB-2017-007/authorlist.pdf}}
% }

% ATLAS reference code, to help ATLAS members to locate the paper
\AtlasRefCode{GROUP-2017-XX}

% ATLAS note number. Can be an COM, INT, PUB or CONF note
% \AtlasNote{ATLAS-CONF-2017-XXX}
% \AtlasNote{ATL-PHYS-PUB-2017-XXX}
% \AtlasNote{ATL-COM-PHYS-2017-XXX}

% Author and title for the PDF file
\hypersetup{pdftitle={ATLAS document},pdfauthor={The ATLAS Collaboration}}

%-------------------------------------------------------------------------------
% Content
%-------------------------------------------------------------------------------
\begin{document}

\maketitle

\tableofcontents

% List of contributors - print here or after the Bibliography.
%\PrintAtlasContribute{0.30}
%\clearpage


%-------------------------------------------------------------------------------
\newpage
\section{Change Log}
\label{sec:change}
%-------------------------------------------------------------------------------
\textbf{Version 0.1}
\begin{itemize}
\item First Draft
\end{itemize}

%-------------------------------------------------------------------------------
%\newpage
%\section{Introduction}
%\label{sec:intro}

\chapter{Introduction}
\label{ch:Introduction}
%Deep thoughts go here.
Over the last 50 years, the Standard Model of particle physics has proven itself time and time again, successfully describing phenomena covering several orders of magnitude in energy as well as describing new particles, such as the Higgs boson, which was finally discovered in 2012.  However, this spectacular resilience is also a source of great frustration.  The Standard Model is known to be incomplete, lacking answers for neutrino masses, the mass of the Higgs boson, dark matter...yet every precision search has yet to poke any new holes in it.  With the Large Hadron Collider (LHC) reaching a new center of mass energy in 2015, the hope is to discover new particles which only become accessible at very high energies.

As a hadron collider, the LHC produces collisions which are dominated by strongly interacting particles which are detected in the form of jets.  Given this very high cross-section, the dijet search is positioned to take advantage of the extraordinary statistics to look for evidence of strongly interacting new resonances across a very large mass range, and hopefully pointing the way for precision searches to explore any excesses that are seen.

This thesis presents the resonance portion of the dijet analysis performed on the combined 2015 and 2016 datasets taken by the ATLAS experiment.\cite{Dijet2017}  The analysis looks at the dijet invariant mass spectrum for evidence of localized excesses which could point to new resonances or other forms of physics beyond the Standard Model.  The analysis uses a model-agnostic methodology to be sensitive to as many possible Beyond the Standard Model (BSM) models as possible, and its results are presented both in terms of limits on selected benchmark models as well as for generic Gaussians which can be used to interpret limits for other theoretical models.

This research paper result was by no means an individual effort, but was built on the efforts of dozens of other researchers, both past and present.  The author's individual contributions included maintenance of the core analysis code, preparation, processing, and dissemination of real and simulated data samples, Wilks' statistical testing, year-to-year dataset validation, and serving as analysis contact and paper editor.  Chapters VII and VIII contain work previously published in Physical Review D, Volume 96, on September 28th, 2017.\cite{Dijet2017}  Chapter IX contains work which has been made public, some of which is awaiting publication.\cite{DijetISR_Resolved}\cite{DijetTLA}\cite{DijetISR}  These works are co-authored with the ATLAS Collaboration, comprising approximately 2900 authors, the full list of which are available in the full publications.

Chapter~\ref{ch:Theory} introduces the Standard Model as well as the benchmark models used in the analysis.  Chapters~\ref{ch:Detector} and~\ref{ch:Calorimetry} discuss the Large Hadron Collider and ATLAS Experiment, with a special focus on the calorimetry systems used to measure jets.  Chapter~\ref{ch:Jets} covers the strong force and how it gives rise to the jets measured in this analysis, while Chapter~\ref{ch:Calibration} discusses how jets are reconstructed and calibrated into final physics objects.  Chapter~\ref{ch:SearchStrategy} covers the search strategy used in this analysis and the sources of uncertainty in the result.  Finally, Chapter~\ref{ch:Results} discusses the finding and limits set on models of new physics while Chapter~\ref{ch:Conclusion} puts these limits in context of other complementary searches as well as the future outlook.


%-------------------------------------------------------------------------------
\newpage
\section{Object Definitions}
\label{sec:objdef}
%-------------------------------------------------------------------------------


%-------------------------------------------------------------------------------
\newpage
\section{Event Selection}
\label{sec:evsel}
%-------------------------------------------------------------------------------

You can find some text snippets that can be used in papers in \texttt{latex/atlassnippets.sty}.
To use them, provide the \texttt{snippets} option to \texttt{atlasphysics}.


%-------------------------------------------------------------------------------
\newpage
\section{Background Estimations}
\label{sec:bkg}
%-------------------------------------------------------------------------------

%-------------------------------------------------------------------------------
\newpage
\section{Neural Network for Signal Background Separation}
\label{sec:neuralnet}
%-------------------------------------------------------------------------------

%-------------------------------------------------------------------------------
\newpage
\section{Analysis}
\label{sec:analysis}
%-------------------------------------------------------------------------------

Place your results here.

% All figures and tables should appear before the summary and conclusion.
% The package placeins provides the macro \FloatBarrier to achieve this.
% \FloatBarrier


%-------------------------------------------------------------------------------
\newpage
\section{Conclusion}
\label{sec:conclusion}
%-------------------------------------------------------------------------------

Place your conclusion here.


%-------------------------------------------------------------------------------
% If you use biblatex and either biber or bibtex to process the bibliography
% just say \printbibliography here
%\printbibliography
% If you want to use the traditional BibTeX you need to use the syntax below.
\bibliographystyle{obsolete/bst/atlasBibStyleWithTitle}
\bibliography{FCNCtqaINT,bib/ATLAS,bib/CMS,bib/ConfNotes,bib/PubNotes}
%-------------------------------------------------------------------------------

%-------------------------------------------------------------------------------
% Print the list of contributors to the analysis
% The argument gives the fraction of the text width used for the names
%-------------------------------------------------------------------------------
\clearpage
The supporting notes for the analysis should also contain a list of contributors.
This information should usually be included in \texttt{mydocument-metadata.tex}.
The list should be printed either here or before the Table of Contents.
\PrintAtlasContribute{0.30}


%-------------------------------------------------------------------------------
\clearpage
\appendix
\part*{Appendices}
\addcontentsline{toc}{part}{Appendices}
%-------------------------------------------------------------------------------

In an ATLAS note, use the appendices to include all the technical details of your work
that are relevant for the ATLAS Collaboration only (e.g.\ dataset details, software release used).
This information should be printed after the Bibliography.
%-------------------------------------------------------------------------------
\section{Introduction}
\label{sec:intro}
%-------------------------------------------------------------------------------
%-------------------------------------------------------------------------------
\section{Introduction}
\label{sec:intro}
%-------------------------------------------------------------------------------

\end{document}
