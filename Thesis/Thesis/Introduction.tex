\chapter{Introduction}
\label{ch:Introduction}

%Deep thoughts go here.\label{sec:bsmFCNC}


The Standard Model of particle physics has proven itself to be an exceptional and resilient theory since the combination of the electromagnetic and weak interactions in 1961\cite{SM1Glashow}.  Further theoretical work combined the Higgs Mechanism\cite{Higgs1,Higgs2} with the electroweak theory\cite{SM2Weinberg, SM3Salam}.  The resiliency of this theoretical model has been tested to further degrees of accuracy over the decades with one of the most recent tests being the experimental confirmation of the Higgs boson in 2012\cite{Higgs3,Higgs4} using the Large Hadron Collider (LHC).  Further precision measurements are ongoing at various experiments at the LHC, including the ATLAS experiment.

However, the Standard Model is known to have flaws and disagreements with nature.  For example, the Standard Model predicts massless neutrinos which is in conflict with experimental observation of neutrino flavor oscillation and does not provide an explanation for dark matter particles. Additionally, the mass of the Higgs boson is light because of large radiative corrections from Standard Model particles which is refered to as the fine-tuning or naturalness problem.  Some theories which attempt to address this naturalness problem are discussed in Section \ref{sec:bsmFCNC}.  While these large gaps in the Standard Model are well known, every precision measurement made has yet to yield any significant new hints toward physics beyond the Standard Model.  

\section{The Standard Model Top Quark}
The top quark was first observed at Fermilab's Tevatron in 1995\cite{TopObs} but the increase in energy and amount of data at the LHC has produced orders of magnitude more top quarks than previously seen, opening up a pathway to precision measurements of the properties of the top quark.  The top quark is the heaviest fundamental particle with a mass of 172.51 $\pm$ 0.27 (stat) $\pm$ 0.42 (syst)\cite{TopMass2017}.  This large mass also means that the top quark lifetime is very short (5 $\times$ 10$^{-25}$ s) and decays before it can hadronize.  This allows the study of its branching ratios and decay modes directly.  The Standard Model predicts that the top quark decays through the charged current mode nearly 100\% of the time, t$\rightarrow$ qW (q = b,s,d)\cite{PDG2018}.   The Standard Model also predicts a rare branching ratio of the top quark through a flavor-changing neutral current (FCNC) process, to a neutral boson (photon, Z boson, Higgs Boson, or gluon) and up-type quark with a heavily suppressed branching ratio on the order of 10$^{-14}$ \cite{2HDM-2}.

\section{Searching for FCNC Top Quark Decays}
Precision measurements are an important litmus test for the Standard Model.  Predicted branching ratios for FCNC processes in top quark decays are far beyond the experimental reach of the LHC and any observation of these decay modes would be a sure sign of new physics.  Branching ratios are an important measurement due to a litany of theories for new physics beyond the Standard Model (BSM).  These BSM theories  such as Minimal Supersymmetric models\cite{MSSM}, R-parity-violating Supersymmetic models\cite{RPVSUSY}, and two-Higgs-doublet models\cite{2HDM} introduce great enhancements to these FCNC branching ratios in the top sector by many orders of magnitude.  Even a null result of a search will set an upper limit on the branching ratio that can assist in ruling out future physical models based on their amount of large top sector enrichment.

This dissertation presents a search for top FCNCs using the entire Run-2 dataset at the LHC, containing combined 2015-2018 datasets taken by the ATLAS experiment totaling 139 fb$^{-1}$ of integrated luminosity taken at $\sqrt{s}$= 13 TeV.  This analysis looks for an excess of events coming from top quark pair-produced events where one top quark decays to the most likely decay mode (a bottom quark and W boson) and the other to an up-type quark (up or charm) and a photon. Chapter \ref{ch:Theory} presents a theoretical background for the Standard Model with a closer view on the usual extensions to include the FCNC vertices.  Following this,  Chapter \ref{ch:LHCDetector} will discuss the LHC and the ATLAS experiment used in the creation of the dataset used in the analysis.  In Chapter \ref{ch:Simulation} the special signal simulation requirements will be presented as well as the common background event simulation methodology.  The search strategy, including the creation of signal, control and validation regions and the training of a neural network, will be examined in Chapter \ref{ch:SearchStrategy}.  Chapter \ref{ch:Results} will discuss the results, and the conclusions drawn from these results will be presented in Chapter \ref{ch:Conclusion}.  Chapters \ref{ch:SearchStrategy} and \ref{ch:Results} include material coauthored with the ATLAS Collaboration.