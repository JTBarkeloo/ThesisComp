
\chapter{Outlook and Conclusions}
\label{ch:Conclusion}
%Deep thoughts go here.
Various theories predicting physics beyond the Standard Model of particle physics are constantly being tested and the Standard Model has stood up to these tests time and time again.  It is imperative to continue probing the Standard Model from every angle to find pieces of the Standard Model that are not completely consistent with experimental observation.  Top quarks provide an ideal avenue to search for these deviations from the expected values in the Standard Model as they are produced in record numbers at the LHC.  This search has set limits on the $t\rightarrow q \gamma$ process which, in turn, helps to limit future theoretical models that predict enhancements to the flavor changing neutral current process.

\section{Comparison with Complementary Searches}

The previous ATLAS search results searching for the production mode diagram (Figure \ref{fig:ProductionMode})\cite{GregorFCNC} suggested a similar result in the up quark channel due to the enhancement achieved by taking advantage of the parton density function of the protons being collided.  While the up quark channel is very competitive, the charm channel limit is weaker, again due to the parton density function.  

The search presented in this dissertation is final state quark independent as the flavor changing neutral current decay should not favor a single light quark final state.  Therefore, the limits achieved searching in the decay mode provide strong bounds on both final states. 

\section{Future Directions}
The prospects of repeating this search throughout the remaining lifetime of the LHC with similar energies and luminosities should provide a statistical benefit to lower the limit further, with the exact amount depending on the amount of statistics and total amount of $t\bar{t}$ pairs the LHC is able to produce.  The High-Luminosity LHC will create a large number of top pair events as the ATLAS experiment will be able to collect more data faster which results in a larger dataset to search for new physics with top quarks.  Further increases in energy for circular colliders such as the LHC also provide a large increase in statistics as the probability to produce heavy particles increases and the $t\bar{t}$ cross section goes up significantly.  The cross section increased by almost a factor of 3 from the LHC Run-1 $\sqrt{s}=8$ TeV to the $\sqrt{s}=13$ TeV and continues to grow with center of mass energy.

Beyond the lifetime of the LHC, precision experiments are expected to be continually performed.  These searches would be performed at a linear collider, for example, the International Linear Collider \cite{Buesser:2013pza} or the Compact Linear Collider (CLIC) \cite{Zarnecki:2018lup}, or various proposed circular colliders such as the Circular Electron Positron Collider (CEPC), Large Hadron electron Collider (LHeC),  and the Future Circular Collider in the electron-positron scenario (FCC-ee) or the hadron-electron scenario (FCC-he)\cite{Benedikt:2015kqj}.   Future searches are expected to be able to push the sensitivities to branching ratios up to two orders of magnitude smaller than presented in this dissertation\cite{Cakir:2018ruj} benefitting from much cleaner datasets (electron-positron colliders operating near the energy required to directly produce $t\bar{t}$ pairs) or a significantly greater amount of data (circular colliders).

\section{Conclusion}

A search has been performed to search for the flavor changing neutral current decay in top quark pair events ($t\bar{t}\rightarrow bl\nu q \gamma$) at the LHC.  This search was performed using the entire Run-2 dataset collected by the ATLAS detector between 2015 and 2018 while the LHC was operating at a center of mass energy of $\sqrt{s}=13$ TeV.  As no signal has been observed, an observed (expected) upper limit on the branching ratio BR($t\rightarrow q \gamma$)$<$ $9.6\times10^{-5}$ $(11.0\times10^{-5})$ and a corresponding upper limit on the cross section $\sigma$($pp\rightarrow t\bar{t} \rightarrow bWq\gamma$) $< 50 (60)$fb have been presented.  This search offers the most stringent limits on the search for FCNC decays in the decay mode using top pair events as well as the best existing limit on the process $t\rightarrow c \gamma$ while being competitive with the production mode search for the process $t\rightarrow u \gamma$.

% BR($t\rightarrow q \gamma$)$<$ $9.6\times10^{-5}$ $(11.0^{+4 .3}_{-3.0}\times10^{-5})$
