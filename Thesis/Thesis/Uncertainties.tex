
\chapter{Systematic Uncertainties}
\label{ch:Uncertainties}
%Deep thoughts go here.
Here thar be uncertainties...probably.

\section{Jet Energy Scale Uncertainty}
The full treatment of jet energy scale uncertainty involves 84 nuisance parameters, the bulk of which come from the bin-to-bin correlations of the various in-situ analyses (Z+jet, $\gamma$+jet, and multijet balance), along with terms covering the eta intercalibration, pile-up, and the behavior of high-\pt jets.  Taking into account all of these variations requires a considerable amount of computing power and is infeasible for this analysis.  Instead, the analysis uses the strongly reduced uncertainties provided by the JetETMiss CP group.  These reduced sets consist of a single eta intercalibration term and three additional nuisance parameters which are combinations of the other 83 parameters.  These parameters are reduced down in four different configurations, and the dataset is tested against all four configurations to determine if the final analysis observables are sensitive to jet correlations.  The final dijet spectrum showed negligible differences between strong reduction sets, and thus any one of them can be used without loss of sensitivity compared to the full set of nuisance parameters.  For the dijet analysis, the uncertainty is dominated by whichever parameter contains the high-\pt term.

\section{Luminosity Uncertainty}
A luminosity uncertainty is applied as a scale factor to the normalization of the various signal samples used in the resonance analysis.  For the combined 2015+2016 dataset, the uncertainty on the luminosity is 3.2\%.  This value was derived from a preliminary calibration of the luminosity scale using the results of the two van der Meer scans performed in August 2015 and May 2016.  In these scans, the x-y beam separation of the two low intensity beams is scanned over, allowing for a measurement of the effective 