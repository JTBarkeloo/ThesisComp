
\chapter{Introduction}
\label{ch:Introduction}
%Deep thoughts go here.
Over the last 50 years, the Standard Model of particle physics has proven itself time and time again, successfully describing phenomena covering several orders of magnitude in energy as well as describing new particles, such as the Higgs boson, which was finally discovered in 2012.  However, this spectacular resilience is also a source of great frustration.  The Standard Model is known to be incomplete, lacking answers for neutrino masses, the mass of the Higgs boson, dark matter...yet every precision search has yet to poke any new holes in it.  With the Large Hadron Collider (LHC) reaching a new center of mass energy in 2015, the hope is to discover new particles which only become accessible at very high energies.

As a hadron collider, the LHC produces collisions which are dominated by strongly interacting particles which are detected in the form of jets.  Given this very high cross-section, the dijet search is positioned to take advantage of the extraordinary statistics to look for evidence of strongly interacting new resonances across a very large mass range, and hopefully pointing the way for precision searches to explore any excesses that are seen.

This thesis presents the resonance portion of the dijet analysis performed on the combined 2015 and 2016 datasets taken by the ATLAS experiment.\cite{Dijet2017}  The analysis looks at the dijet invariant mass spectrum for evidence of localized excesses which could point to new resonances or other forms of physics beyond the Standard Model.  The analysis uses a model-agnostic methodology to be sensitive to as many possible Beyond the Standard Model (BSM) models as possible, and its results are presented both in terms of limits on selected benchmark models as well as for generic Gaussians which can be used to interpret limits for other theoretical models.

This research paper result was by no means an individual effort, but was built on the efforts of dozens of other researchers, both past and present.  The author's individual contributions included maintenance of the core analysis code, preparation, processing, and dissemination of real and simulated data samples, Wilks' statistical testing, year-to-year dataset validation, and serving as analysis contact and paper editor.  Chapters VII and VIII contain work previously published in Physical Review D, Volume 96, on September 28th, 2017.\cite{Dijet2017}  Chapter IX contains work which has been made public, some of which is awaiting publication.\cite{DijetISR_Resolved}\cite{DijetTLA}\cite{DijetISR}  These works are co-authored with the ATLAS Collaboration, comprising approximately 2900 authors, the full list of which are available in the full publications.

Chapter~\ref{ch:Theory} introduces the Standard Model as well as the benchmark models used in the analysis.  Chapters~\ref{ch:Detector} and~\ref{ch:Calorimetry} discuss the Large Hadron Collider and ATLAS Experiment, with a special focus on the calorimetry systems used to measure jets.  Chapter~\ref{ch:Jets} covers the strong force and how it gives rise to the jets measured in this analysis, while Chapter~\ref{ch:Calibration} discusses how jets are reconstructed and calibrated into final physics objects.  Chapter~\ref{ch:SearchStrategy} covers the search strategy used in this analysis and the sources of uncertainty in the result.  Finally, Chapter~\ref{ch:Results} discusses the finding and limits set on models of new physics while Chapter~\ref{ch:Conclusion} puts these limits in context of other complementary searches as well as the future outlook.
